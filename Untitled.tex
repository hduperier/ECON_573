% Options for packages loaded elsewhere
\PassOptionsToPackage{unicode}{hyperref}
\PassOptionsToPackage{hyphens}{url}
%
\documentclass[
]{article}
\usepackage{amsmath,amssymb}
\usepackage{lmodern}
\usepackage{iftex}
\ifPDFTeX
  \usepackage[T1]{fontenc}
  \usepackage[utf8]{inputenc}
  \usepackage{textcomp} % provide euro and other symbols
\else % if luatex or xetex
  \usepackage{unicode-math}
  \defaultfontfeatures{Scale=MatchLowercase}
  \defaultfontfeatures[\rmfamily]{Ligatures=TeX,Scale=1}
\fi
% Use upquote if available, for straight quotes in verbatim environments
\IfFileExists{upquote.sty}{\usepackage{upquote}}{}
\IfFileExists{microtype.sty}{% use microtype if available
  \usepackage[]{microtype}
  \UseMicrotypeSet[protrusion]{basicmath} % disable protrusion for tt fonts
}{}
\makeatletter
\@ifundefined{KOMAClassName}{% if non-KOMA class
  \IfFileExists{parskip.sty}{%
    \usepackage{parskip}
  }{% else
    \setlength{\parindent}{0pt}
    \setlength{\parskip}{6pt plus 2pt minus 1pt}}
}{% if KOMA class
  \KOMAoptions{parskip=half}}
\makeatother
\usepackage{xcolor}
\IfFileExists{xurl.sty}{\usepackage{xurl}}{} % add URL line breaks if available
\IfFileExists{bookmark.sty}{\usepackage{bookmark}}{\usepackage{hyperref}}
\hypersetup{
  pdftitle={Econ 573 Assignment 3},
  pdfauthor={Harvey Duperier},
  hidelinks,
  pdfcreator={LaTeX via pandoc}}
\urlstyle{same} % disable monospaced font for URLs
\usepackage[margin=1in]{geometry}
\usepackage{color}
\usepackage{fancyvrb}
\newcommand{\VerbBar}{|}
\newcommand{\VERB}{\Verb[commandchars=\\\{\}]}
\DefineVerbatimEnvironment{Highlighting}{Verbatim}{commandchars=\\\{\}}
% Add ',fontsize=\small' for more characters per line
\usepackage{framed}
\definecolor{shadecolor}{RGB}{248,248,248}
\newenvironment{Shaded}{\begin{snugshade}}{\end{snugshade}}
\newcommand{\AlertTok}[1]{\textcolor[rgb]{0.94,0.16,0.16}{#1}}
\newcommand{\AnnotationTok}[1]{\textcolor[rgb]{0.56,0.35,0.01}{\textbf{\textit{#1}}}}
\newcommand{\AttributeTok}[1]{\textcolor[rgb]{0.77,0.63,0.00}{#1}}
\newcommand{\BaseNTok}[1]{\textcolor[rgb]{0.00,0.00,0.81}{#1}}
\newcommand{\BuiltInTok}[1]{#1}
\newcommand{\CharTok}[1]{\textcolor[rgb]{0.31,0.60,0.02}{#1}}
\newcommand{\CommentTok}[1]{\textcolor[rgb]{0.56,0.35,0.01}{\textit{#1}}}
\newcommand{\CommentVarTok}[1]{\textcolor[rgb]{0.56,0.35,0.01}{\textbf{\textit{#1}}}}
\newcommand{\ConstantTok}[1]{\textcolor[rgb]{0.00,0.00,0.00}{#1}}
\newcommand{\ControlFlowTok}[1]{\textcolor[rgb]{0.13,0.29,0.53}{\textbf{#1}}}
\newcommand{\DataTypeTok}[1]{\textcolor[rgb]{0.13,0.29,0.53}{#1}}
\newcommand{\DecValTok}[1]{\textcolor[rgb]{0.00,0.00,0.81}{#1}}
\newcommand{\DocumentationTok}[1]{\textcolor[rgb]{0.56,0.35,0.01}{\textbf{\textit{#1}}}}
\newcommand{\ErrorTok}[1]{\textcolor[rgb]{0.64,0.00,0.00}{\textbf{#1}}}
\newcommand{\ExtensionTok}[1]{#1}
\newcommand{\FloatTok}[1]{\textcolor[rgb]{0.00,0.00,0.81}{#1}}
\newcommand{\FunctionTok}[1]{\textcolor[rgb]{0.00,0.00,0.00}{#1}}
\newcommand{\ImportTok}[1]{#1}
\newcommand{\InformationTok}[1]{\textcolor[rgb]{0.56,0.35,0.01}{\textbf{\textit{#1}}}}
\newcommand{\KeywordTok}[1]{\textcolor[rgb]{0.13,0.29,0.53}{\textbf{#1}}}
\newcommand{\NormalTok}[1]{#1}
\newcommand{\OperatorTok}[1]{\textcolor[rgb]{0.81,0.36,0.00}{\textbf{#1}}}
\newcommand{\OtherTok}[1]{\textcolor[rgb]{0.56,0.35,0.01}{#1}}
\newcommand{\PreprocessorTok}[1]{\textcolor[rgb]{0.56,0.35,0.01}{\textit{#1}}}
\newcommand{\RegionMarkerTok}[1]{#1}
\newcommand{\SpecialCharTok}[1]{\textcolor[rgb]{0.00,0.00,0.00}{#1}}
\newcommand{\SpecialStringTok}[1]{\textcolor[rgb]{0.31,0.60,0.02}{#1}}
\newcommand{\StringTok}[1]{\textcolor[rgb]{0.31,0.60,0.02}{#1}}
\newcommand{\VariableTok}[1]{\textcolor[rgb]{0.00,0.00,0.00}{#1}}
\newcommand{\VerbatimStringTok}[1]{\textcolor[rgb]{0.31,0.60,0.02}{#1}}
\newcommand{\WarningTok}[1]{\textcolor[rgb]{0.56,0.35,0.01}{\textbf{\textit{#1}}}}
\usepackage{graphicx}
\makeatletter
\def\maxwidth{\ifdim\Gin@nat@width>\linewidth\linewidth\else\Gin@nat@width\fi}
\def\maxheight{\ifdim\Gin@nat@height>\textheight\textheight\else\Gin@nat@height\fi}
\makeatother
% Scale images if necessary, so that they will not overflow the page
% margins by default, and it is still possible to overwrite the defaults
% using explicit options in \includegraphics[width, height, ...]{}
\setkeys{Gin}{width=\maxwidth,height=\maxheight,keepaspectratio}
% Set default figure placement to htbp
\makeatletter
\def\fps@figure{htbp}
\makeatother
\setlength{\emergencystretch}{3em} % prevent overfull lines
\providecommand{\tightlist}{%
  \setlength{\itemsep}{0pt}\setlength{\parskip}{0pt}}
\setcounter{secnumdepth}{-\maxdimen} % remove section numbering
\ifLuaTeX
  \usepackage{selnolig}  % disable illegal ligatures
\fi

\title{Econ 573 Assignment 3}
\author{Harvey Duperier}
\date{2022-10-04}

\begin{document}
\maketitle

\hypertarget{part-i}{%
\section{Part I}\label{part-i}}

\hypertarget{question-2}{%
\subsection{Question 2}\label{question-2}}

\textbf{2a).} The lasso, relative to least squares, is: \emph{iii.} Less
flexible and hence will give improved prediction accuracy when its
increase in bias is less than its decrease in variance. This is because
the lasso can yield a reduction in variance when compared to least
squares in exchange for a small increase in bias, consistently
generating more accurate predictions, also making it easier to
interpret, making iii the correct choice.

\textbf{2b).} The ridge regression, relative to least squares, is: iii.
Less flexible and hence will give improved prediction accuracy when its
increase in bias is less than its decrease in variance. This is because,
similarly to lasso, the ridge regression can yield a reduction in
variance, when compared to least squares, in exchange for a small
increase in bias. The relationship between λ and variance and bias is
important: when it increases, the flexibility of the ridge regression
decreases which causes decreased variance, but increased bias, making
iii the correct choice again.

\textbf{2c).} Non-linear methods, relative to least squares, is: ii.
More flexible and hence will give improved prediction accuracy when its
increase in variance is less than its decrease in bias. Contrasting to
ridge and lasso methods, non-linear methods work in the opposite way,
giving increased prediction accuracy when a decrease in bias gives way
to an increase in variance, making ii the correct choice.

\hypertarget{question-3}{%
\subsection{Question 3}\label{question-3}}

\textbf{3a).} As we increase s from 0, the training RSS will: iv.
Steadily decrease. As we begin to increase s from 0, all beta's will
increase from 0 to their least square estimate values. The training RSS
for beta's at 0 will be the maximum and trend downward to the original
least squares RSS; therefore, iv is the correct choice.

\textbf{3b).} As we increase s from 0, the test RSS will: ii. Decrease
initially, and then eventually start increasing in a U shape. When s=0
and all beta's are 0, the model is extremely simple and because of that,
has a high test RSS. Beginning to increase s, beta s will begin to
assume non-zero values and the model begins to fit better, so test RSS
originally decreases. Eventually, beta s will approach their OLS values,
and as they begin to overfit the training data, test RSS will begin to
increase again, forming a U shape and making ii the correct choice.

\textbf{3c).} As we increase s from 0, the variance will: iii. Steadily
Increase. When s=0, the model basically predicts a constant and has
almost no variance, but as we increase s, the model includes more Betas,
and their values will begin to increase. As the values of Betas become
increasingly more dependent on training data, the variance will steadily
increase, making iii the correct choice.

\textbf{3d).} As we increase s from 0, the (squared) bias will: iv.
Steadily Decrease. As we stated in the previous example, when s=0, the
model basically predicts a constant, so the prediction is far from the
actual value, and (squared) bias is high. As we increase s from 0
though, more Beta's become non-zero, and the model continues to fit the
training data better, thus making bias steadily decrease, and proving iv
to be the correct choice.

\textbf{3e).} As we increase s from 0, the irreducible error will: v.
Remain Constant. Irreducible error is model dependent and therefore
increasing s from 0 will not change it, making it remain constant and
proving v to be the best choice.

\hypertarget{question-10}{%
\subsection{Question 10}\label{question-10}}

\begin{Shaded}
\begin{Highlighting}[]
\FunctionTok{summary}\NormalTok{(cars)}
\end{Highlighting}
\end{Shaded}

\begin{verbatim}
##      speed           dist       
##  Min.   : 4.0   Min.   :  2.00  
##  1st Qu.:12.0   1st Qu.: 26.00  
##  Median :15.0   Median : 36.00  
##  Mean   :15.4   Mean   : 42.98  
##  3rd Qu.:19.0   3rd Qu.: 56.00  
##  Max.   :25.0   Max.   :120.00
\end{verbatim}

\hypertarget{including-plots}{%
\subsection{Including Plots}\label{including-plots}}

You can also embed plots, for example:

\includegraphics{Untitled_files/figure-latex/pressure-1.pdf}

Note that the \texttt{echo\ =\ FALSE} parameter was added to the code
chunk to prevent printing of the R code that generated the plot.

\end{document}
